%\documentclass[12pt,notitlepage,aps,pra,longbibliography,nofootinbib,tightenlines]{revtex4}
%\documentclass[12pt,notitlepage,longbibliography,nofootinbib,tightenlines]{revtex4-1}
%\documentclass[12pt,notitlepage,longbibliography,nofootinbib,tightenlines]{revtex4-1}

%\documentclass[12pt,a4]{revtex4}
\documentclass[12pt]{article}
%\documentclass[11pt, twocolumn]{article}

%\usepackage{epsf}
\usepackage{amsmath}
\usepackage{color}
\usepackage{natbib}
%\usepackage{cite}

\RequirePackage{amsmath}
\RequirePackage{amssymb}
\RequirePackage{amsthm}
%\RequirePackage{algorithmic}
%\RequirePackage{algorithm}
%\RequirePackage{theorem}
%\RequirePackage{eucal}
\RequirePackage{color}
\RequirePackage{url}
\RequirePackage{mdwlist}

\RequirePackage[all]{xy}
\CompileMatrices
\RequirePackage{hyperref}
\RequirePackage{graphicx}
%\RequirePackage[dvips]{geometry}


\makeatletter
\newcommand{\pushright}[1]{\ifmeasuring@#1\else\omit\hfill$\displaystyle#1$\fi\ignorespaces}
\newcommand{\pushleft}[1]{\ifmeasuring@#1\else\omit$\displaystyle#1$\hfill\fi\ignorespaces}
\makeatother


\begin{document}

\title{The Surface Betrays What Lies Beneath}

\author{Simon Burton}
%\affiliation{Centre for Engineered Quantum Systems, School of Physics, The University of Sydney}

\date{\today}

%\begin{abstract}
%We make a compendium of many uses of a simple idea.
%Perhaps it is the only really great calculation.
%\end{abstract}

\maketitle


\def\Complex{\mathbb{C}}
\def\Z{\mathbb{Z}}
\def\Ham{\mathcal{H}}
\def\Pauli{\mathcal{P}}
\def\Spec{\mbox{Spec}}
\def\Proveit{{\it (Proof??)}}
\def\GL{\mathrm{GL}}
\def\half{\frac{1}{2}}
\def\Stab{S}

%\def\morph


%%%%%%%%%%%%%%%%%%%%%%%%%%%%%%%%%%%%%%%%%%%%%%%%%%%%%%%%%%%%%%%%%%%%%%%%%%%%%%%
%
%%%%%%%%%%%%%%%%%%%%%%%%%%%%%%%%%%%%%%%%%%%%%%%%%%%%%%%%%%%%%%%%%%%%%%%%%%%%%%%
%

%\section{Introduction}

The so-called ``Generalized Distributive Law''
\cite{Aji2000}
explores the ramifications of the innocent looking
formula $ab+ac = a(b+c).$
These include such gems as the fast Fourier transform
and belief propagation.
We can also think of multiplication by $a$ as a function $f$
and rewrite this equation as 
$$
    f(b)\hat{+}f(c)=f(b+c) \ \ \ \ \ \ \ \ \ \ \ \mbox{(M)}
$$
This is the equation for a homomorphism, or representation.
It says we can represent addition (on the right-hand side)
as some other kind of addition, $\hat{+}.$
%The simple distributive law states that we can zoom in or out

\section{Fast Fourier transform}

% Apparently invented by Gauss:
% http://www.cis.rit.edu/class/simg716/Gauss_History_FFT.pdf


The discrete Fourier transform of a sequence $\{x_j\}_0^{N-1}$
is given by
$$
%    y_k = \sum_{j=0}^{N-1} x_j e^{2\pi i k \frac{j}{N}}, \ \ k=0,...,N-1.
    y_k = \sum_{j=0}^{N-1} x_j \omega^{kj}, \ \ k=0,...,N-1.
$$
where $\omega=e^{2\pi i/N}.$
If $N$ is a composite number $N=N_1 N_2$ then we can re-express this sum as
\begin{align*}
    y_k &= \sum_{j_1=0}^{N_1-1} \sum_{j_2=0}^{N_2-1} x_{j_1 N_2 + j_2}
            \omega^{k (j_1 N_2 + j_2)}, \ \ k=0,...,N-1. \\
        &= \sum_{j_1=0}^{N_1-1}
            \omega^{k {j_1 N_2}}
            \sum_{j_2=0}^{N_2-1} x_{j_1 N_2 + j_2}
            \omega^{k {j_2}}, \ \ k=0,...,N-1.
\end{align*}
Note there are two applications of (M) here. ((three??))
The first turns a sum into a product: $\omega^{a+b}=\omega^a \omega^b.$
Then we can apply vanilla distributive law to refactor the summation.
If we pre-compute the exponentials, this leads to a reduction in
the number of operations required, hence the adjective ``fast''.

Displaying this graphically.


\section{Dynamic programming}

Dijkstra's algorithm

\section{Huygen's principle}

Path integrals, calculus of variations

Integration by parts involves a crucial use of the distributive law.

Stoke's theorem ? % Chern-Simons ?

\section{Fast marching}

Eikonal equation

\section{Automatic differentiation}

forward and reverse

\section{Logic: cut-elimination}

Natural deduction, Gentzen's sequent calculus..

\section{Combinatorics...}

path counting, transfer matrix...
generating functions?

\section{Renormalization}

works in 1d...

\section{Homology/Co-Homology}

Forward and reverse


\bibliography{refs}{}
\bibliographystyle{abbrv}


\end{document}


