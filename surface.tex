%\documentclass[12pt,notitlepage,aps,pra,longbibliography,nofootinbib,tightenlines]{revtex4}
%\documentclass[12pt,notitlepage,longbibliography,nofootinbib,tightenlines]{revtex4-1}
%\documentclass[12pt,notitlepage,longbibliography,nofootinbib,tightenlines]{revtex4-1}

%\documentclass[12pt,a4]{revtex4}
\documentclass[12pt]{article}
%\documentclass[11pt, twocolumn]{article}

%\usepackage{epsf}
\usepackage{amsmath}
\usepackage{color}
\usepackage{natbib}
%\usepackage{cite}

\RequirePackage{amsmath}
\RequirePackage{amssymb}
\RequirePackage{amsthm}
%\RequirePackage{algorithmic}
%\RequirePackage{algorithm}
%\RequirePackage{theorem}
%\RequirePackage{eucal}
\RequirePackage{color}
\RequirePackage{url}
\RequirePackage{mdwlist}

\RequirePackage[all]{xy}
\CompileMatrices
\RequirePackage{hyperref}
\RequirePackage{graphicx}
%\RequirePackage[dvips]{geometry}


\makeatletter
\newcommand{\pushright}[1]{\ifmeasuring@#1\else\omit\hfill$\displaystyle#1$\fi\ignorespaces}
\newcommand{\pushleft}[1]{\ifmeasuring@#1\else\omit$\displaystyle#1$\hfill\fi\ignorespaces}
\makeatother


\begin{document}

\title{The surface betrays what lies beneath}

\author{Simon Burton}
%\affiliation{Centre for Engineered Quantum Systems, School of Physics, The University of Sydney}

\date{\today}

%\begin{abstract}
%We make a compendium of many uses of a simple idea.
%Perhaps it is the only really great calculation.
%\end{abstract}

\maketitle


\def\Complex{\mathbb{C}}
\def\Z{\mathbb{Z}}
\def\Ham{\mathcal{H}}
\def\Pauli{\mathcal{P}}
\def\Spec{\mbox{Spec}}
\def\Proveit{{\it (Proof??)}}
\def\GL{\mathrm{GL}}
\def\half{\frac{1}{2}}
\def\Stab{S}

%\def\morph


%%%%%%%%%%%%%%%%%%%%%%%%%%%%%%%%%%%%%%%%%%%%%%%%%%%%%%%%%%%%%%%%%%%%%%%%%%%%%%%
%
%%%%%%%%%%%%%%%%%%%%%%%%%%%%%%%%%%%%%%%%%%%%%%%%%%%%%%%%%%%%%%%%%%%%%%%%%%%%%%%
%

%\section{Introduction}

The so-called ``Generalized Distributive Law''
\cite{Aji2000}
explores the ramifications of the innocent looking
formula $ab+ac = a(b+c).$
These include such gems as the fast Fourier transform
and belief propagation.
We can also think of multiplication by $a$ as a function $f$
and rewrite this equation as 
$$
    f(b)\hat{+}f(c)=f(b+c) \ \ \ \ \ \ \ \ \ \ \ \mbox{(M)}
$$
This is the equation for a homomorphism, or representation.
It says we can represent addition (on the right-hand side)
as some other kind of addition, $\hat{+}.$
%The simple distributive law states that we can zoom in or out

\section{Fast Fourier transform}

% Apparently invented by Gauss:
% http://www.cis.rit.edu/class/simg716/Gauss_History_FFT.pdf


The discrete Fourier transform of a sequence $\{x_j\}_0^{N-1}$
is given by
$$
%    y_k = \sum_{j=0}^{N-1} x_j e^{2\pi i k \frac{j}{N}}, \ \ k=0,...,N-1.
    y_k = \sum_{j=0}^{N-1} x_j \omega^{kj}, \ \ k=0,...,N-1.
$$
where $\omega=e^{2\pi i/N}.$
If $N$ is a composite number $N=N_1 N_2$ then we can re-express this sum as
\begin{align*}
    y_k &= \sum_{j_1=0}^{N_1-1} \sum_{j_2=0}^{N_2-1} x_{j_1 N_2 + j_2}
            \omega^{k (j_1 N_2 + j_2)}, \ \ k=0,...,N-1. \\
        &= \sum_{j_1=0}^{N_1-1}
            \omega^{k {j_1 N_2}}
            \sum_{j_2=0}^{N_2-1} x_{j_1 N_2 + j_2}
            \omega^{k {j_2}}, \ \ k=0,...,N-1.
\end{align*}
Note there are two applications of (M) here. ((three??))
The first turns a sum into a product: $\omega^{a+b}=\omega^a \omega^b.$
Then we can apply vanilla distributive law to refactor the summation.
If we pre-compute the exponentials, this leads to a reduction in
the number of operations required, hence the adjective ``fast''.

Displaying this graphically.

\section{Belief propagation}

Economics

Bethe Anatz (solves Heisenberg in 1D), cavity method.
Is there a forward \& reverse?

\section{Dijkstra's algorithm}

%\def\P{\mathcal P}
\def\Z{\mathbb Z}

Let $G$ be a finite graph: $G = (V, E).$
%Choose a POSET $\P$ with unique 
For $x,y\in V$ Write $x\sim y$ when there
is an edge between $x$ and $y.$ % or when $x=y.$
%Let $\P=\mathbb Z \cup \{\infty\}$ be the ordered set of integers with a top element.
Define the set of \emph{states} as %``height'' functions
$A = \{ f | f : V \to \Z \}.$
This is a partially ordered set, using
the pointwise order:
$f\le g$ iff $f(x)\le g(x)\ \forall x\in V.$

Define the \emph{message passing} function
$T:A\to A$ by
$$(Tf)(x) = \min\{f(x), \min_{y\sim x} f(y)+1\}.$$

{\bf Lemma:} there exists $n$ such that $T^{n+1} = T^n.$

{\bf Proof:}
$T$ is monotonic non-increasing on $ A.$
For any $f\in A$ we have $g\le f$
where $g$ is a constant function defined by $g(x)=\min_y f(y).$
Therefore the set $\{f, Tf, T^2f,...\}$ is finite.
\emph{XXX need to use $G$ finite}
\qed

%Let $n$ be the minimum $n$ such that $T^{n+1}=T^n.$ 

\def\Fix{\mathrm{Fix}}

Any $f\in  A$ such that $Tf=f$ is called \emph{stationary.}
The set of such states we denote $\Fix(T).$
%Any constant function  is stationary.
%The functions furthest away from being stationary
%are the $f$ that have $T^nf\ne T^{n-1}f.$
%We call these the \emph{points} of $T.$
%We would like to identify these functions with
%the vertices of the graph, but there are too many such $f$.
A state $f$ is stationary iff
$f(x)\le f(y)+1,\ \forall x\sim y.$
Direct computation shows that
%set of stationary states is convex:
$\Fix(T)$ is convex:
given $f$ and $g$ both stationary,
and $\alpha\in[0, 1]$ then $\alpha f + (1-\alpha) g$
is stationary.

We define an equivalence relation on 
states $f$ and $g$ by 
declaring $f\sim g$ when $f-g$ is a constant state. 
This quotient map is a projection $P,$
and $P(\Fix(T))$ is now a compact convex subset of
a finite dimensional real vector space.
Furthermore, this set is the intersection
of a finite set of closed half-spaces and
so has finitely many extreme points.
For each edge $x\sim y$ of the graph
we get a closed half space
$f(x)\le f(y)+1.$
This half space survives the action of $P$
because we can rewrite it as $f(x)-f(y)\le 1.$

The extreme points of $P(\Fix(T))$ are bijectively
identified with the vertices of $G.$
The identification works as follows.
Let $f$ be such an extreme point.
$f$ will saturate a set of inequalities,
ie. $\{f(x)-f(y)=1\}$
This set must touch every vertex of $G.$
There are no loops in this set, and so we
have a spanning tree.... the minimum of $f$
is the vertex of $G$ that we seek...


\section{Dynamic programming}

Linear programming, simplex method (duality)

Approxiamte Dynamic programming (Reinforcement learning)
% https://webdocs.cs.ualberta.ca/~bowling/papers/07adprl-dualrl.pdf

Bellman's equation a.k.a. Hamilton-Jacobi equations

\section{Huygen's principle}

Path integrals, calculus of variations

Integration by parts involves a crucial use of the distributive law.

Stoke's theorem ? % Chern-Simons ? AdS-CFT correspondence ?

Green's function...

\section{Fast marching}

Eikonal equation

\section{Automatic differentiation}

forward and reverse

\section{Logic: cut-elimination}

Natural deduction, Gentzen's sequent calculus..

\section{Combinatorics...}

path counting, transfer matrix...
generating functions? 
creation/annihilation operators?
denotational semantics?

\section{Renormalization?}

works in 1d... dual is zoom-in ?

\section{Homology/Co-Homology?}

Forward and reverse

\section{Type inference?}

 Haskell can infer the result type of a function from its argument (forward), or the argument type from the result (backward)

\bibliography{refs}{}
\bibliographystyle{abbrv}


\end{document}


