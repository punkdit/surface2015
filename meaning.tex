%\documentclass[12pt,notitlepage,aps,pra,longbibliography,nofootinbib,tightenlines]{revtex4}
%\documentclass[12pt,notitlepage,longbibliography,nofootinbib,tightenlines]{revtex4-1}
%\documentclass[12pt,notitlepage,longbibliography,nofootinbib,tightenlines]{revtex4-1}

%\documentclass[12pt,a4]{revtex4}
\documentclass[11pt]{article}
%\documentclass[11pt, twocolumn]{article}


% xelatex:
\usepackage{fontspec}
\defaultfontfeatures{Ligatures=TeX}
%\usepackage[small,sf,bf]{titlesec}

%\setromanfont{DejaVu Serif}
%\setromanfont{Droid Serif}

%\setromanfont{Gentium} % nice! a bit fluffy
\setromanfont{Gentium Book Basic} % more bold

%\setromanfont{Noto Serif} % a bit thick
%\setromanfont{Utopia}


%\usepackage{epsf}
\usepackage{amsmath}
\usepackage{color}
\usepackage{natbib}
\usepackage{tikz-cd}
%\usepackage{cite}

\RequirePackage{amsmath}
\RequirePackage{amssymb}
\RequirePackage{amsthm}
%\RequirePackage{algorithmic}
%\RequirePackage{algorithm}
%\RequirePackage{theorem}
%\RequirePackage{eucal}
\RequirePackage{color}
\RequirePackage{url}
\RequirePackage{mdwlist}

\RequirePackage[all]{xy}
\CompileMatrices
\RequirePackage{hyperref}
\RequirePackage{graphicx}
%\RequirePackage[dvips]{geometry}


\makeatletter
\newcommand{\pushright}[1]{\ifmeasuring@#1\else\omit\hfill$\displaystyle#1$\fi\ignorespaces}
\newcommand{\pushleft}[1]{\ifmeasuring@#1\else\omit$\displaystyle#1$\hfill\fi\ignorespaces}
\makeatother


\begin{document}

%\title{The surface betrays what lies beneath}
%\title{Show me the morphism}
%\title{When the surface betrays what lies beneath}
%\title{Path-integrals are everywhere}
%\title{The meaning is in the arrows between objects, not in the objects themselves}
%\title{Design patterns in message passing}
%\title{Size, distributivity, homomorphism:\\
%    when the surface betrays what lies beneath}

%9/2/2017
%========
%
%Theme: 
%Wandering Towards a Goal: How can mindless mathematical laws give rise to aims and intention?

\title{Meaning and Compositionality}

\author{Simon Burton}
%\affiliation{Centre for Engineered Quantum Systems, School of Physics, The University of Sydney}

\date{\today}

%\begin{abstract}
%We make a compendium of many uses of a simple idea.
%Perhaps it is the only really great calculation.
%\end{abstract}

\maketitle


\def\N{\mathbb N}
\def\Z{\mathbb Z}
\def\R{\mathbb R}
\def\C{\mathbb C}
\def\Sets{\mathbb B}
\def\Expect{\mathbb E}
\def\Ind{\mathbb I}
\def\Complex{\mathbb{C}}
\def\GL{\mathrm{GL}}
\def\half{\frac{1}{2}}
\def\todo#1{\emph{(XXX #1 XXX)}}
\def\tensor{\otimes}
\def\tr{\mbox{tr}}
\def\det{\mbox{det}}


%%%%%%%%%%%%%%%%%%%%%%%%%%%%%%%%%%%%%%%%%%%%%%%%%%%%%%%%%%%%%%%%%%%%%%%%%%%%%%%
%
%%%%%%%%%%%%%%%%%%%%%%%%%%%%%%%%%%%%%%%%%%%%%%%%%%%%%%%%%%%%%%%%%%%%%%%%%%%%%%%
%


The physicists answer: energy minimization, the ball ``wants'' to
go to the bottom of the hill.

Here we discuss examples, mostly simple, rather than abstract theory.

All theories have a ``gauge'' symmetry. For example the theory that
says there are five ``things'' has a symmetry, we can permute the five
things.

Logic: zero, negatives, fractions
Is there a ``half'' or is it just a something waiting
to divide another number by two ?
%to find a symmetry in a theory? 

Compositionality or distributivity
AI, physics, mathematics, economics

Returning to the physicists answer, we take a somewhat
more subtle approach of Lagrangian mechanics, which is
essentially calculus of variations. Now the ball somehow
``knows'' much more: all possible paths, among which it
chooses a stationary such path.
The key inside the calculus of variations involves integrating
by parts against a test function. It's distributivity again.
Huygen's principle.

All of this becomes much deeper when we consider complex
representations, ie. quantum physics.
Measurement is distributive.

If you posit such a front, it ``creates'' the appearance (reflection)
of intention. 
But the Huygen's wavefront is illusory: this is what compositionality says. 
There is no ``one'' having aims and intentions: it only appears to
be so when the wavefront is presumed.


\bibliography{refs}{}
\bibliographystyle{abbrv}


\end{document}

