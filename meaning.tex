

9/2/2017
========

Theme: 
Wandering Towards a Goal: How can mindless mathematical laws give rise to aims and intention?

Essay:
Meaning and Compositionality

The physicists answer: energy minimization, the ball ``wants'' to
go to the bottom of the hill.

Here we discuss examples, mostly simple, rather than abstract theory.

All theories have a ``gauge'' symmetry. For example the theory that
says there are five ``things'' has a symmetry, we can permute the five
things.

Logic: zero, negatives, fractions
Is there a ``half'' or is it just a something waiting
to divide another number by two ?
%to find a symmetry in a theory? 

Compositionality or distributivity
AI, physics, mathematics, economics

Returning to the physicists answer, we take a somewhat
more subtle approach of Lagrangian mechanics, which is
essentially calculus of variations. Now the ball somehow
``knows'' much more: all possible paths, among which it
chooses a stationary such path.
The key inside the calculus of variations involves integrating
by parts against a test function. It's distributivity again.
Huygen's principle.

All of this becomes much deeper when we consider complex
representations, ie. quantum physics.
Measurement is distributive.

If you posit such a front, it ``creates'' the appearance (reflection)
of intention. 
But the Huygen's wavefront is illusory: this is what compositionality says. 
There is no ``one'' having aims and intentions: it only appears to
be so when the wavefront is presumed.




